\chapter{2014年2月23日}
\section{wifi定位系统—Android端实现}
\subsection{将航位推算集成到百度地图}
\begin{enumerate}
    \item
    进行实验测试
    \begin{enumerate}
    \item
    记录两点之间的经纬度
    \item
    记录步数
    \item
    用航位推算软件记录位置信息
    \end{enumerate}
    \item
    得出测试结果(经纬度和DeadReckoning软件记录的位置的函数关系)
    \item
    代码合成
\iffalse
    \item
    进行定位算法的几要点:
    \begin{enumerate}
    \item
    第一:收集多组SS样本,并且从一个基站使用一个样本均值。
    \item
    第二:为了确定位置和方向,确定用哪一组数据
    \begin{enumerate}
    \item
    一:加入离线状态下的数据进行考虑
    \item
    二:信号传播建模(develop a model that accounts for both free-space loss and loss)
    \end{enumerate}
    \item
     第三:need a metric and a search methodology to compare multiple locations and pick the one that best matches the observed signal strength
    \end{enumerate}
    \item
    对于以下三种方法:The empirical method performs significantly better than both of the other methods
    \begin{enumerate}
    \item
    First:empirical method
    \item
    Second:strongest base station
    \item
    Third:random method
    \end{enumerate}
    \item
    Unlike the basic analysis where we only considered the single nearest neighbor in signal space,we now consider k nearest neighbors, for various values of k.averaging the coordinates of the neighbors may yield an estimate that is closer to the user’s true location
\fi
\end{enumerate}
\subsection{代码阅读整理}
E:/JNHomework/wifilocation/location/codes/zhangyi/vt-ece-3574-dead-reckoning-read-only
\begin{enumerate}
\item
所需包或类
\begin{enumerate}
\item
edu.vt.dr.GLAndSensorsActivity
\item
edu.vt.dr.utilities
\end{enumerate}
\item
如何获取位置信息
\begin{enumerate}
\item
首先GLAndSensorsActivity.java中onSensorChanged方法调用了包utilities下SensorUtil类中的routeEvent方法,通过map类型的mHandlers匹配到其中在AccelerationHandler中实现了EventHandler中的service接口,然后就知道其中调用了LocationUtil类中的accelerationUpdate方法,其中就包含了位置更新算法。
\item
然后我们的目标就很简单了:就是将一个Activity个一个包提取出来。
将Activity中实现SensorEventListener接口,然后类似的初始化和传感器监听。当我们从包中获取到位置信息后就直接更新百度地图的显示(重新定位)
\end{enumerate}
\end{enumerate}

